% !TeX program = xelatex

\newcommand{\No}{\textnumero}

%%% Здесь выбираются необходимые графы
\documentclass[russian,utf8,pointsection,nocolumnxxxi,nocolumnxxxii,12pt]{eskdtext}
\usepackage{fontspec}
\defaultfontfeatures{Mapping=tex-text} % Для того чтобы работали стандартные сочетания символов ---, --, << >> и т.п.

%%% Что бы работал eskdx и некоторые другие пакеты LaTeX
\usepackage{xecyr}

%%% Для работы шрифтов
\usepackage{xunicode,xltxtra}


\DeclareMathSizes{12}   {12}   {12}    {12}

%%% Для работы с русскими текстами (расстановки переносов, последовательность комманд строго обязательна)
\setmainfont{GOST_type_A}
\setromanfont{GOST_type_A} 
\setsansfont{GOST_type_A} 
\setmonofont{GOST_type_A} 


%%% Для работы со сложными формулами
\usepackage{amsmath}
\usepackage{amssymb}

%%% Что бы использовать символ градуса (°) - \degree
\usepackage{gensymb}

\usepackage{longtable}
\usepackage{makecell}
\usepackage{caption}
\usepackage{multirow}

\usepackage{ragged2e}
\usepackage{blindtext}

%%% Для переноса составных слов
%\XeTeXinterchartokenstate=1
\XeTeXcharclass `\- 24
\XeTeXinterchartoks 24 0 ={\hskip\z@skip}
\XeTeXinterchartoks 0 24 ={\nobreak}

%%% Ставим подпись к рисункам. Вместо «рис. 1» будет «Рисунок 1»
\addto{\captionsrussian}{\renewcommand{\figurename}{Рисунок}}
%%% Убираем точки после цифр в заголовках
\def\russian@capsformat{%
  \def\postchapter{\@aftersepkern}%
  \def\postsection{\@aftersepkern}%
  \def\postsubsection{\@aftersepkern}%
  \def\postsubsubsection{\@aftersepkern}%
  \def\postparagraph{\@aftersepkern}%
  \def\postsubparagraph{\@aftersepkern}%
}

\usepackage[compact]{titlesec} 
\titlespacing{\section}{10mm}{10mm}{10mm}

% Автоматически переносить на след. строку слова которые не убираются в строке
\sloppy

%%% Для вставки рисунков
\usepackage{graphicx}

%%% Для вставки интернет ссылок, полезно в библиографии
\usepackage{url}

%%% Подподразделы(\subsubsection) не выводим в содержании
\setcounter{tocdepth}{2}

%%% Каждый раздел (\section) с новой страницы
%\let\stdsection\section
%\renewcommand\section{\newpage\stdsection}

%%% Название документа
\ESKDtitle{ Электроды покрытые металлические марки "<marka>" }
\ESKDdocName{ Паспорт }

\ESKDauthor{ <auth> }
\ESKDchecker{ <check> }
\ESKDnormContr{ <norm> }
\ESKDapprovedBy{ <app> }

\newcommand\ESKDauthorTitle{<authTitle>}
\newcommand\ESKDcheckerTitle{<checkTitle>}
\newcommand\ESKDapprovedByTitle{<appTitle>}

%%% Для титульника
\ESKDtitleApprovedBy{ \ESKDapprovedByTitle }{ \ESKDtheApprovedBy }
\ESKDtitleDesignedBy{ Разработал: \\ \ESKDauthorTitle }{ \ESKDtheAuthor }
\ESKDtitleDesignedBy{ Проверил: \\ \ESKDcheckerTitle }{ \ESKDtheChecker }

\renewcommand{\ESKDtheTitleFieldX}{
Судиславский район, Костромская область, д. Текотово \\ \ESKDtheYear г.
}

\renewcommand{\textsuperscript}[1]{\raisebox{0.8ex}{\scalebox{0.66}{#1}}}
\renewcommand{\textsubscript}[1]{\raisebox{-0.4ex}{\scalebox{0.66}{#1}}}

%\ESKDdepartment{ Ведомство }
\ESKDcompany{ ООО "Судиславский завод сварочных материалов" }
\ESKDclassCode{ <okp> }
\ESKDsignature{ ПС СЗСМ <npart> }
\ESKDdate{ <date> }

\begin{document}

%%% Делаем титульник
\maketitle

%%% Делаем содержание
\tableofcontents
\pagebreak[4]

\section{ Основные сведения об изделии }

\begin{longtable}{|p{5cm}|p{10cm}|}
\hline 
Изготовитель: & Общество с ограниченной ответственностью "Судиславский завод сварочных материалов" (ООО «СЗСМ»)  \\
\hline 
Адрес места нахождения: & Российская Федерация, Костромская область, Судиславский район, деревня Текотово, Промзона-1, дом 2  \\
\hline 
Телефон, электронная почта: & (49433) 2-55-56, info@szsm-mail.ru \\
\hline 
Наименование изделия: & Электроды покрытые металлические \\
\hline 
Нормативный документ (НД): & <tustr> \\
\hline 
Назначение изделия:  & <descr> \\
\hline 
Марка: & <marka> \\
\hline 
Диаметр, мм: & <diam> \\
\hline 
Длина, мм: & <long> \\
\hline 
Тип по ГОСТ 9467: & <type> \\
\hline 
Условное обозначение: & \vspace{-2.5mm}  $ <frac> $ \vspace{2.5mm}\\
\hline 
Номер партии: & <npart> \\
\hline 
Масса партии, кг: & <massa> \\
\hline 
Дата изготовления: & <dat> г. \\
\hline 
<sert>
\end{longtable}

\section{Технические данные}

\begin{longtable}{|p{5cm}|p{10cm}|}
\hline 
Марка проволоки по ГОСТ 2246: & <provol>  \\
\hline 
Вид покрытия: & <grp>  \\
\hline 
Коэффициент массы покрытия, \%: & <kfmp> \\
\hline 
<plav>
Химический состав наплавленного металла, \%: &  
\vspace{-2.5mm} 
<chem>
\vspace{2.5mm} 
\\
\hline 
<mech>
\end{longtable}

\section{Срок хранения и гарантии изготовителя}

Изготовитель гарантирует соответствие электродов требованиям НД при условии соблюдения правил хранения и транспортировки. Гарантийный срок от даты продажи электродов — <warr> месяцев. 
Срок годности (хранения) не ограничен при условии соблюдения правил хранения и транспортировки.

\section{Заметки по эксплуатации и хранению}

Содержание влаги в покрытии электродов перед использованием не должно превышать <vl>\%.
При необходимости, перед использованием, необходимо прокалить при температуре <proc>. Допускается повторная прокалка электродов не более 3 раз.

<amp>

Электроды следует хранить в сухих отапливаемых помещениях при температуре не ниже плюс 15 °С в условиях, предохраняющих их от загрязнения, увлажнения и механических повреждений.

\section{Требования безопасности}

Организация и проведение сварочных работ должны осуществляется в соответствии с Федеральным законом от 17.07.1999 N 181-ФЗ "Об основах охраны труда в Российской Федерации" и Федеральным законом от 28 декабря 2013 г. N 426-ФЗ "О специальной оценке условий труда". \par
Сварочные работы должны выполняться в соответствии с требованиями следующих документов:
\begin{itemize}
\item ГОСТ 12.3.003-86 Система стандартов безопасности труда. Работы электросварочные. Требования безопасности
\item ГОСТ 12.3.004-75 Система стандартов безопасности труда. Термическая обработка металлов. Общие требования безопасности

\item Правила по охране труда при выполнении электросварочных и газосварочных работ» (утв. Приказом Министерства труда и социальной защиты Российской Федерации от 23 декабря 2014 года №1101н), 
\item Правила по охране труда при эксплуатации электроустановок (утв. приказом Министерства труда и социальной защиты Российской Федерации от 24 июля 2013 года N 328н) 
\item Правила технической эксплуатации электроустановок потребителей (утв. приказом Минэнерго России от 13 января 2003 года N 6)
Санитарные правила при сварке, наплавке и резке металлов (утв. Главным государственным санитарным врачом СССР 05.03.1973 N 1009-73)
\end{itemize}
\par
Для защиты от искр и брызг расплавленного металла, вредных излучений сварочной дуги в видимой, ультрафиолетовой и инфракрасной областях, влаги и других вредных факторов производственной среды сварщики должны применять индивидуальные средства защиты, включающие, как минимум, щиток сварщика, спецодежду, рукавицы или перчатки, обувь.
\par
При выполнении сварочных работ в условиях повышенной опасности поражения электрическим током в целях электробезопасности сварщики должны применять диэлектрические перчатки, галоши и коврики.
\par
Все средства индивидуальной защиты должны быть исправными, обеспечивать необходимый уровень защиты и соответствовать требованиям технического регламента Таможенного союза «О безопасности средств индивидуальной защиты» (ТР ТС 019/2011).
\par
Средства индивидуальной защиты должны применяться в соответствии с указаниями изготовителя.
Запрещается использование поврежденных средств индивидуальной защиты или средств индивидуальной защиты с истекшим сроком службы (сроком годности).

\section{Свидетельство о приемке}

\begin{tabular}{ccc}
 $ \frac {\text{Электроды покрытые металлические}}{\text{\tiny наименование изделия}} $
 &
 $\frac {<frac>}{\text{\tiny обозначение}} $ 
 & 
  $\frac {\text{<npart>}}{\text{\tiny заводской номер партии}} $ 
 \end{tabular}
 \vspace{5mm} 
 \par
Изготовлены и приняты в соответствии с требованиями <tustr> и признаны годными для эксплуатации.
 \par
 \vspace{10mm} 

\begin {flushright}
\begin{tabular}{ccccc}
МП \hspace{5 mm}
&
 $ \frac {\text{\ESKDcheckerTitle}}{\text{\tiny должноть}} $
 &
 $ \frac {\text{}}{\text{\tiny \hspace{5 mm} личная подпись \hspace{5 mm}}} $
 & 
  $\frac {\text{\ESKDtheChecker}}{\text{\tiny расшифровка подписи}} $ 
  & 
  $\frac {\text{\ESKDtheYear.\ESKDtheMonth.\ESKDtheDay}}{\text{\tiny год, месяц, число}} $ 
 \end{tabular}
\end{flushright}

\section{Сведения об утилизации}

Утилизация отходов электродов должна проводиться путем сдачи в специализированные организации, осуществляющие деятельность по сбору и использованию отходов V класса опасности.

\end{document}
